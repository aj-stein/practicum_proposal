\documentclass{jdf}

\begin{document}
\title{Practicum Proposal: \\ Mutual Monitoring in the Cloud}
\author{Alexander Stein \\ \href{mailto:astein38@gatech.edu}{astein38@gatech.edu}}

\maketitle
\thispagestyle{fancy}

\section{Problem Statement}

Cloud computing infrastructure is essentially ubiquitous, but adoption is not without challenges. Cloud service providers must cater to customers in regulated sectors, with complex cybersecurity frameworks, with high barriers to entry. One barrier is ongoing evaluation of the provider's cybersecurity posture, resulting in centralized bureaucracies.

Are these bureaucracies an optimal solution or a last resort that failed to keep pace with cloud technology? If the latter, is there a better way?

Many regulatory frameworks for cybersecurity require frequent, periodic reviews of a provider's security posture. Such reviews have significant compounding costs for the cloud service providers, contracted independent auditors, and customers who build upon them. The resulting processes Despite the proliferation of automation platforms and security analysis tools, there is often manual data exchange and coordination between their staff. These manual processes motivate bureaucracies. And those bureaucracies prefer methods of qualitative analysis, not quantitative ones. I will confirm this claim with a critical analysis of FedRAMP's Continuous Monitoring Program \citeyear[p.~14]{fedramp_auth_playbook25} and their scaling challenges as a key example. I will then use this analysis to identify requirements and test the feasibility of an alternative to the FedRAMP model of continuous monitoring: mutual continuous monitoring without a centralized regulatory body.

\section{Choice of Problem}

The cybersecurity of cloud services poses many challenges, but the inefficiency of continuous monitoring has systemic impact on the macroeconomics and risk modeling for heavily interconnected systems built on cloud services. FedRAMP is a highly visible and representative example that other regulatory frameworks emulate, so any improvement or optimization will yield significant improvement to cloud service adoption across regulated industries.

\subsection{Economic Impacts}

Although FedRAMP is a highly visible cloud security authorization program, there is limited precise data on the costs and economic impact for providers, auditors, and customer agencies. However, industry estimates significant costs for all these stakeholders, even when considering global expenditure on cloud services.

Gartner estimates that global spending on cloud infrastructure in 2024 was \$595.7 billion dollars \citeyear{gartner24}. The think tank CSIS estimates that the United States government spent seventeen of its total \$130 billion dollar IT budget in 2024 on cloud services alone \citeyear[p.~1]{csis25}. Although there is not complete compliance with FedRAMP requirements in government agencies, the goal of the FedRAMP Authorization Act that mandates FedRAMP's program, including the continuous monitoring, is ongoing oversight over the cloud building blocks of this seventeen billion dollar investment. And continuous monitoring is a sizable component of this investment.

FedRAMP processes require specialized tools operated by dedicated staff, from providers, auditors, and often the customer agencies. Analysts at stackArmor estimate that an initial authorization costs a provider \$250,000 to \$750,000 dollars, of which \$100,000 to \$400,000 alone is for continuous monitoring activities \citeyear{stackarmor24}. Given a conservative estimate, any improvement or optimization can benefit all stakeholders in reducing \$42,600,000 in spend by 426 services currently authorized, but potentially a much larger sum.

\subsection{Cybersecurity Impacts}

Even with all this investment, the staff from cloud service providers, auditors, and agency customers experience strategic and operational bottlenecks for heavily interconnected cloud services, increasing ambiguity in a holistic view of cybersecurity posture in real-world composite systems. 

Firstly, a centralized review process finalized by a small number of FedRAMP staff constitutes a single point of failure. As FedRAMP documents, cloud providers, auditors, and agency customers must use a centralized wiki site, USDA's connect.gov, and coordinate out of band with FedRAMP staff for final review \citeyear[pp.~3,14]{fedramp_auth_playbook25}. Paradoxically, providers and auditors get no guarantees for the cybersecurity posture of this system where they store data for FedRAMP's reviewers. The system and underlying processes also limit sharing with auditors and customers focused on other regulatory frameworks. They rely on reciprocity guarantees to justify the use of FedRAMP authorization and continuous monitoring.

The impacts of manually curated data from FedRAMP's continuous monitoring extend beyond its stakeholders. Interrelated regulatory frameworks depend upon it. Given FedRAMP's rigorous review process, especially continuous monitoring, many providers and their auditors use artifacts from FedRAMP authorization for equivalency, or reciprocity, for deliverables as evidence for other regulatory frameworks, especially in the defense, commercial, and finance sectors in the United States. Therefore, any optimization in FedRAMP's processes has second order effects on the quality, quantity, and speed of cloud security review methodologies across industry.

\section{Expected Deliverables}

To best research alternatives to popular centralized models for continuous modeling, I propose the list of deliverables below, in addition to the final report summarizing their outcome. 

\begin{enumerate}
    \item a critical analysis of the centralized FedRAMP model for continuous monitoring
    \item an architecture specification for mutual continuous monitoring
    \item prototype code for transparency services for mutual continuous monitoring
    \item a quantitative measurement framework for cloud service security to use in this architecture
\end{enumerate}

\bibliographystyle{apacite}
\bibliography{references.bib}

\end{document}
